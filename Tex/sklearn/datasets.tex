\chapter{datasets\label{datasets}}
\begin{table}
    \centering
    \caption{datasets}
    \begin{tabular}{llllll}
        \hline
        \nameref{fetch20newsgroups} & \nameref{fetchlfwpeople} & \nameref{makefriedman1} & \nameref{makefriedman2} & \nameref{makefriedman3} & \nameref{makecircles} \\
        \nameref{makeblobs}         &                                                                                                                                \\
        \hline
    \end{tabular}
\end{table}
\section{make\_blobs\label{makeblobs}}
\section{make\_friedman1\label{makefriedman1}}
\verb|make_friedman1(n_samples=100, n_features=10, *, noise=0.0, random_state=None)|

Inputs \verb|X| are independent features uniformly distributed on the interval $[0, 1]$. The output \verb|y| is created according to the formula:
\begin{equation}
    y = 10\sin(\pi x_1x_2)+20(x_3-\frac{1}{2})^2+10x_4+5x_5+Gaussian~Noise(0,\sigma)
\end{equation}

A synthetic data set called \textit{Friedman-1}, originally created by Jerome Friedman in 1991 to explore how well his new multivariate adaptive regression splines
(MARS) algorithm was fitting high-dimensional data.

This data set was carefully generated to evaluate a regression method's ability to
only pick up true feature dependencies in the data set and ignore others.

\begin{table}[H]
    \centering
    \caption{make\_friedman1主要参数}
    \begin{tabularx}{\textwidth}{llX}
        \hline
        Properties & Names                                              & Descriptions                                                        \\
        \hline
        Parameters & \verb|n_samples: int, default=100|                 & The number of samples.                                              \\
        Parameters & \verb|n_features: int, default=10|                 & The number of features. Should be at least 5.                       \\
        Parameters & \verb|noise: float, default=0.0|                   & The standard deviation of the gaussian noise applied to the output. \\
        Returns    & \verb|X: ndarray of shape (n_samples, n_features)| & The input samples.                                                  \\
        Returns    & \verb|y: ndarray of shape (n_samples,)|            & The output values.                                                  \\
        \hline
    \end{tabularx}
\end{table}
\section{make\_friedman2\label{makefriedman2}}
\section{make\_friedman3\label{makefriedman3}}
\section{make\_circles\label{makecircles}}
\section{fetch\_20newsgroups\label{fetch20newsgroups}}
\section{fetch\_lfw\_people\label{fetchlfwpeople}}